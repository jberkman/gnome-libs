\documentclass{article}
\begin{document}
\title{Design Guidelines to make Gnome language bindings friendly}
\author{Andreas Kostyrka, andreas@ag.or.at\\
        Marius Vollmer, mvo@zagadka.ping.de}
\maketitle
\tableofcontents
\section{Rationale for this document}
The author of this document has been writing the TOM bindings of
Gnome/Gtk.

In doing so, I've noticed that there are certain constructs and
designing habits with Gnome that make it difficult to wrap the
resulting Gnome code for usage in any other language that
C.\footnote{This includes probably also C++, with the exception, that
the C++ programmer can always include C code with his program, while
this is not an option with other languages.}
I presume that this is not by intend, but by not knowing.\footnote{The
language bindings authors seem to be a minority of the Gnome Hackers.}

This document should help the average Gnome hacker to focus upon his
work and still to produce wrappable code.

This document has been written with TOM in mind, but probably applies
to most other languages too. Should some language binding author have
an additional problem, just send me\footnote{andreas@ag.or.at} an
email.

\section{Differences between C and other languages}

There are certain differences between C, which is basically a very
lowlevel language, and other languages that are usually higher level.

\subsection{Memory management}

C allows the programmer to do practically anything. Other languages do
not have this flexibility.

For example, TOM has the following memory management scheme:
\begin{itemize}
\item memory for TOM objects is garbage collected. memory allocated
      with malloc is not. TOM objects that hold pointers to such C
      language memory blocks should clear it up in the dealloc method
      which is called by the garbage collector.
      The implication here is, that a TOM object doesn't have a fine
      grained control about the time of deallocation, which is
      probably true for all garbage collected languages.
\item 
        In TOM, an object is always handled by reference; think `pointer
        to struct' if you like.  Objects reside on the heap: they can not
        reside on the stack or within another object.

      This means that it is not possible to declare a static struct array to
      describe things in TOM, instead such an array has to build
      dynamically. Again this is probably true for many languages:
      struct array parameters like GnomeUIInfo that are a convenience
      in C, require quite a lot of work to get working with any other language.
\end{itemize}

\subsection{Functions}

Language bindings usually are not able to provide dynamically new C
language functions. In some compiled language like ObjC, C++, and even
TOM this is possible with lesser or greater pains, but interpreted
languages are out of luck.\footnote{No, generating dynamically
function stubs as data and making it executable is not really good
portable programming, and makes also for fun debugging.}

\section{Pitfalls in C}

Here I want to describe some things that are quite possible in C, but
are difficult or impossible to wrap in other languages.

\subsection{variable number of arguments}
Also called the ``The \dots\ problem''. It is not possible to to call
such function in a general fashion in C. Witness, that in C, there are
special versions of printf that take an va\_arg as an argument,
instead of using the standard printf.

The proposed solution is to provide primarily array arguments, and if
doomed important enough to provide convenience functions taking a
variable number of arguments. Example for a current problem in Gnome:
{\tt gnome\_dialog\_new()}.

\subsection{memory ownership}
In C you have 100\% full control how and which memory
    is used. As noted earlier, not so in many languages that are often
    garbage collected.
    So if at all possible, a pointer passed in should be deep copied.
    Example: TOM bindings for GnomeUIInfo arrays. These bindings force the
    TOM programmer to retain the object somewhere where it will not be
    collected because collecting the TOM object frees also the C structure
    assembled, which could have \emph{bad} influence on Gnome.

    Basically for most languages one has to assume, that they are
    garbage collected, and that pointers in GtkObjects don't protect
    host language objects. 

\subsection{usage patterns}
When designing an widget make clear how certain structs are intended
to be used. That is why GtkObjects are rather good (as they can be
dealt with once in general in OO languages\footnote{In tomgtk for
example GtkObject deals with all the proxing stuff, and child classes
don't need to reconsider how to proxy themselves.}), while GdkStructs are
problematic.

If you have to use non-OO structs, follow one usage pattern: Either
copy the struct, or store the pointer. Catering for two different
usages makes the bindings only more difficult than necessary.
An example for correct usage is {\tt GdkColor}, which seems always to
be copied. Try if at all to provide functions to copy the objects, and
do other basic functionality: Some language could want to create
temporaries, and managing them as references to another instance of
the object complicates things drastically.

\subsection{Requirements for callbacks}
Callbacks must always carry one freely setable pointer
argument. Callbacks should also have a DestroyNotify.

This requirements come from the fact that it is not acceptable to
expect
the user to create a new function for every menu item, as it may be
completely impossible to do so in some languages.

Usually language bindings map all callbacks (at least of one kind) to a
    given function, and this function uses the pointer argument to
    know which binding in the host language is meant.  As the data
    areas for method binding are probably dynamically allocated, a
    DestroyNotify is a "good thing"(tm) to prevent memory leaks. 

Example with problems: {\tt gnome\_mdi\_child\_set\_menu\_template} that uses
    gnomeapp\_create\_menus\_with\_data. This is basically
    non usable from anything else than C/C++/ObjC\footnote{And in C++/ObjC it
    also means some contortions in design probably.}.

\subsection{Subclassing}
At the moment I believe there is
    no language wrapping for subclassing a GtkObject class. To wrap
    this process is by
    design difficult, as it involves structs that point to callbacks
    without the needed extra parameter and other niceties :( 
    So don't expect full blown GtkObject subclassing in anything else
    than C in the near future.

    Do not
    force the user of your widget to subclass some object to use your
    code. You can make it easier by subclassing but always leave a
    possibility to do it without full blown subclassing. To explain
    what is probably manageable from all languages: 
\begin{verbatim}
C: GnomeMDIChild                    TOM: GnomeMDIChild 
                                    \- MyChild 
\end{verbatim}
So on the C side {\tt MyChild} objects are
    represented by {\tt GnomeMDIChilds}. Callbacks are handled in such an
    setup not by overwriting function pointers in the class struct (as
    in {\tt gnome-hello-7-mdi MyChild} class), but by connecting signals to
    these slots.

\section{Last comments}
As I had TOM in mind when I wrote this, I'd like to hear comments
from other language bindings authors and also comments from gnome-libs
designers in how far my ideas are doable.

Also, English is not my native language, so any corrections in this
area are welcome too.

Anyway, I've just been told that someone else want's to say something,
too, so I'm handing the mike over to, \dots, err, excuse me what was
your name again?

\section{More stuff}

Cough, cough, hello, my name is Marius Vollmer and I've done the Guile
bindings so I thought I just step up here and, well, ok, I just drone
on, then\dots

I'd like to bring up two points and illustrate them with some sample
code.
\begin{itemize}
\item The importance of destroy notification,
\item Making reference counting and a tracing GC cooperate.
\end{itemize}

The sample code will implement a simple struct that is reference
counted and can initiate a callback.  Here it is:
\begin{quote}
\begin{verbatim}
typedef struct _foo foo;
typedef void (*foo_callback) (foo *, int n, void *data);

struct _foo {
  int ref_count;
  foo *peer;

  foo_callback callback_func;
  void *callback_data;
  GtkDestroyNotify callback_notify;
};

foo *
foo_new ()
{
  foo *f = alloc_one_foo ();
  f->ref_count = 1;
  f->peer = NULL;
  f->callback_func = NULL;
  f->callback_notify = NULL;
}

void
foo_ref (foo *f)
{
  f->ref_count++;
}

void
foo_unref (foo *f)
{
  f->ref_count--;
  if (f->ref_count == 0)
    {
      if (callback_notify)
        f->callback_notify (f->callback_data);
      if (f->peer)
        foo_unref (f->peer);
      free_one_foo (f);
    }
}

void
foo_set_peer (foo *f, foo *p)
{
  if (p)
    foo_ref (p);
  if (f->peer)
    foo_unref (f->peer);
  f->peer = p;
}

void
foo_set_callback (foo *f, foo_callback func, void *data,
                  GtkDestroyNotify notify)
{
  if (f->callback_notify)
    f->callback_notify (f->data);
  f->callback_func = func;
  f->callback_data = data;
  f->callback_notify = notify;
}

static void
foo_invoke_callback (foo *f, int n)
{
  if (f->callback_func)
    f->callback_func (f, n, f->callback_data);
}

void
foo_trace_refs (foo *f, void (*tracer)(void *, void *), void *data)
{
  if (f->peer)
    tracer (f->peer, data);
}
\end{verbatim}
\end{quote}

This is a lot of code but there are reasons for it.  Most of them are
already included in Andreas' text above but I like to repeat them
anyway.

The first thing you might notice is that you need to support the
callback\_data argument for the callback functions.  This allows the
user of your code the ability to pass additional data to the callback
function.  Generally this is used to somehow specify the `environment'
in which the callback function should run.  Using global variables for
this is not suitable for this because there might be any number of
callbacks waiting to be activated and they all need their own private
piece of state.  In general, using global state makes your program
`non-reentrant', which is to be avoided.

Actually, providing this callback\_data mechanism is not only useful to
language binding writers, but also to any other user of your code.  It
is a standard practice, and for good reasons, too.

It might be instructive to study how a typical binding for a
interpreted language would make use of callback\_data.  The functions
in that interpreted language are generally not represented by pointers
to C functions, of course, so they can't be used as the callback\_func
directly.  The language binding will provide a `helper' function that
invokes the interpreted function in the appropriate manner.  The
callback\_data will carry all information that the helper functions
needs to do this, including the interpreted function itself.  If it
were not for the callback\_data `backdoor' the binding would have to
provide a different helper function for every interpreted function.
This is not possible.

What's more, the information that the helper function needs to do its
job might sit in dynamically allocated memory.  The binding code needs
to know when this memory can be freed.  It could try to `second-guess'
your code and free it when it connects a new callback or when a foo
structure is finally freed.  This is impossible to get right for the
general case, however, because the binding code has no guarantee that
it is the only one using the structs.  The only safe way is for the
foo code itself to provide precise notifications when a callback\_data
value goes out of business.  That is what the callback\_notify
function is for.  A particular callback\_notify function is always
associated with a particular callback\_data value.  Whenever the foo
code can guarantee that it will never use this particular
callback\_data value again, it will call its associated
callback\_notify function with the value.

This is all fairly traditional stuff and is useful not only for
language bindings but also makes for generally nice library
interfaces.  The next point is a little bit more involved.  It has to
do with the foo\_trace\_refs function.

It has todo with a memory management technique that your average C
program is not likely to use, but every self-respecting high-level
language has it.  Of course, I'm talking about a `tracing garbage
collector'.  Such a thing is able to detect which objects of your
running program are in use, and which are garbage.  The garbage is
collected and reused for new objects.  This is much like the reference
counting mechanism, but the keyword here is `tracing'.  A tracing GC
is not fooled by cycles of objects like reference counting is.  For
example, when you have two foo structs that each point to one another
with their peer field but are not referenced from other parts of the
program, the are not collected by the reference counting scheme
although they are garbage.

A tracing GC would be able to detect that they are able because it
would actually look into the objects and `trace' the reference each
has to other objects.

The goal here is not to teach the old reference counting dog new
tricks.  The foo structs with their ref counts were not designed to
form cycles of garbage and it would do little good to try to detects
these cycles when using them from a high level language.  Afterall,
this would subtly change the semantics of the foo code and might
actually cause problems for code can't deal with this change.

The goal is to prevent the `cycles of garbage' from leaking into the
domain of the high-level language and ruin \textbf{its} memory
management.

These things can indeed happen.  Remember that callback\_data?  It is
a generic piece of data that the foo struct stores on behalf of the
binding code.  While it is being stored in that object (that is, until
the callback\_notify is called), the binding code has to protect the
value from being collected by its own tracing GC.  But chances are
that the callback\_data value contains references to other values from
the high-level language, so they need to be protected, too.  The next
step is that these referenced values contain references to foo
structs, and these foo structs in turn contain a reference (in a peer
field) to the original foo structs whose callback\_data we are talking
about.  That is, we have formed a cycle that goes halfway thru objects
from the high-level language, and halfway thru foo structs.  The foo
structs are only reference counted, and therefore the binding code is
not able to follow it.  Thus we have a cycle that is potential garbage
but we can't figure that out.

This is not an unlikely situation with guile-gtk.  The callback\_data
would be a `closure', that is, a procedure together with its dynamic
environment.  The environment is very likely to contain all sorts of
references to other stuff including the very objects that the
procedure is connected to as a callback.

The existence of the foo\_trace\_refs function allows the binding code
to find the cycle and collect the objects if necessary.  So please
provide one.  If you don't (and nobody does right now), you will only
produce memory leaks and not core dumps, but still.

\end{document}
